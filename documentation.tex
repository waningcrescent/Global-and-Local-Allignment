\documentclass{article}
\usepackage{graphicx}
\usepackage{ragged2e}

\title{BIO213 – Introduction to Quantitative Biology 
ASSIGNMENT-1
}
\author{Shriya Verma 2021490 }


\begin{document}

\maketitle

\section{Question. 1}
There are two sequences given in the question : GATGCGCAG and GGCAGTA. We are expected to perform global allignment on the given sequences with the following scoring scheme: 
                            \\ \centerline{Match = +2}
                            \\ \centerline{Mismatch = -3}
                            \\ \centerline{Gap = -1}

\subsection{Bidimensional Array obtained in the computation of the Needleman-Wunsch Algorithm }
Using Python and Dynamic Programming to implement the Needleman-Wunsch Algorithm, the bidimensional array obtained is : 
\\
\\
\centering\includegraphics[scale = 0.4]{arrayq1}
\\

\hfill \break
\subsection{Can there be more than one optimal alignments? }
It is possible to get more than one optimal alignment depending on the path taken while tracing the matrix back. Multiple alignments can have the same score. Therefore, it is important to list all the best possible alignments of a given global alignment.
\hfill \break

\\ \centering\includegraphics[scale = 0.2]{traceback.png}

\subsection{Optimal Alignments of the given sequences having maximum score = 4  }

\\ \centering\includegraphics[scale = 0.5]{alignq1}


\RaggedRight\section{Question. 2}
Show Modification of the Results if the Scoring scheme is changed to:
                            \\ \centerline{Match = +2}
                            \\ \centerline{Mismatch = -1}
                            \\ \centerline{Gap = -3}
\subsection{Bidimensional Array obtained in the computation of the Needleman-Wunsch Algorithm }
The matrix obtained is different from the matrix obtained in Question 1, As at iteration, the maximum value for a cell is selected. These values are heavily dependant on the scoring scheme used. The values obtained by gap/matchmismatch will change depending on the scoring scheme, and therefore, the maximum value filled in each cell is also subject to change, implying that the matrix will be different and so will the traceback.
\\
\hfill \break
\text {Iteration}: \quad F(i, j)=\max \left\{\begin{array}{c} 

F(i-1, j) + gap penalty\\ 
F(i, j-1) + gap penalty\\ 
F(i-1, j-1)+match\left(x_{i}, y_{j}\right) 
\end{array}\right. \nonumber
\\
\hfill \break
\hfill \break
\centering\includegraphics[scale = 0.5]{array2}
\RaggedRight\section{Question. 3}
Now, we are required to perform local allignment of the same sequences :  GATGCGCAG and GGCAGTA using the scoring scheme defined below:
                            \\ \centerline{Match = +2}
                            \\ \centerline{Mismatch = -1}
                            \\ \centerline{Gap = -3}
 \subsection{Bidimensional Array obtained in the computation of the Needleman-Wunsch Algorithm }                           
Local alignment is done using Smith-Waterman Algorithm, The initialization step differs from the Needleman-Wunsch Algorithm here as the first row and column are initialized with zeroes unlike the gap penalty used in Needleman-Wunsch Algorithm. The iteration step also differs as only non-negative values are taken to fill the matrix : 
\hfill \break
\text {Iteration}: \quad F(i, j)=\max \left\{\begin{array}{c} 
0\\
F(i-1, j) + gap penalty\\ 
F(i, j-1) + gap penalty\\ 
F(i-1, j-1)+match\left(x_{i}, y_{j}\right) 
\end{array}\right. \nonumber
\\
\hfill \break
The matrix obtained is : 
\hfill \break
\\ \centering\includegraphics[scale = 0.5]{array3}
\\The score for a local alignment is the maximum value in the matrix and can be present at any position.
\subsection{Optimal Alignments of the given sequences having maximum score = 8 }
\\ \centering\includegraphics[scale = 0.5]{align3.png}

\RaggedRight\section{Question. 4}
\subsection{Changes required in the program to perform local rather than
global pairwise sequence alignment}
\begin{enumerate}
    

\item The first change was the initialization of the matrix:
\\For Global Alignment, The matrix is initialized by assigning gap penalties to all cells in the first row/column
\\For Local Alignment, The matrix is initialized by assigning value 0 to all cells in the first row/column

\item Comparison with zero while filling matrix using :

\\  \centerline{\textbf{max(0,value)}}


\\As in the Smith-Waterman Algorithm, the minimum value of any cell is 0.

\item The final change was the initialization of the traceback, While in global alignment, the traceback always starts from the bottom-right-most cell.
\\In local allignment, the traceback starts from the cell having the maximum value.
The code for local is required to iterate through the entire matrix to find the maximum value in the matrix and it's position before calling the traceback function on it. I also had to stop the traceback as soon as a cell value of '0' is reached, contrary to stopping on reaching matrix[0][0].
\item Fundamentally the differences in Global and local alignment are : 
\hfill \break
Global:
\\ 
\\ \centerline{
\begin{array}{ll} 
\text { Initialization }: & F(i, 0) = gappenalty*i\\ 
& F(0, j)=gappenalty*j
\end{array} \nonumber}
\\ 
\hfill \break

\centerline{
\text {Iteration}: \quad F(i, j)=\max \left\{\begin{array}{c}
F(i-1, j) + gap penalty\\ 
F(i, j-1) + gap penalty\\ 
F(i-1, j-1)+match\left(x_{i}, y_{j}\right) 
\end{array}\right. \nonumber
\\}
\hfill \break

\centerline{\text{Termination : Bottom Right} \nonumber}


\hfill \break
Local:
\\
\\
\centerline{
\begin{array}{ll} 
\text { Initialization }: & F(i, 0)=0 \\ 
& F(0, j)=0 
\end{array} \nonumber}
\hfill \break

\\ \centerline{
\text {Iteration}: \quad F(i, j)=\max \left\{\begin{array}{c}
0\\
F(i-1, j) + gap penalty\\ 
F(i, j-1) + gap penalty\\ 
F(i-1, j-1)+match\left(x_{i}, y_{j}\right) 
\end{array}\right. \nonumber
\\}
\hfill \break

\centerline{\text{Termination : max. value in the matrix} \nonumber}

\end{enumerate}
\end{document}
